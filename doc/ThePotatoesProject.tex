%%%%%%%%%%%%%%%%%%%%%%%%%%%%%%%%%%%%%%%%%%%%%%%%%%%%%%%%%%%%%%%%%%%%%%%%%%%%%%%%%%%%%%%%%%%%%%%%%%%%%%%%%%%%%%%
% LFA Project 2017-2018 G13
% Inês Justo, Luís Moura, Maria Lavoura, Pedro Teixeira
% Objectivo: Escrever um documento que descreva a linguagem (instruções existentes e o seu signicado; exemplos de programas; etc.)
\documentclass{report}
\usepackage[T1]{fontenc} % Fontes T1
\usepackage[utf8]{inputenc} % Input UTF8
%\usepackage[backend=biber, style=ieee]{biblatex} % para usar bibliografia
\usepackage{csquotes}
\usepackage[portuguese]{babel} %Usar língua portuguesa
\usepackage{blindtext} % Gerar texto automaticamente
\usepackage[printonlyused]{acronym}
\usepackage{hyperref} % para autoref
\usepackage{graphicx}
\usepackage{listings}
\usepackage{color}
\usepackage{booktabs}
\usepackage{multirow}
\usepackage{geometry}
\geometry{a4paper, margin=1in}

\definecolor{codegreen}{rgb}{0,0.6,0}
\definecolor{codegray}{rgb}{0.5,0.5,0.5}
\definecolor{codepurple}{rgb}{0.58,0,0.82}
\definecolor{backcolour}{rgb}{0.95,0.95,0.92}

%Code listing style named "mystyle"
\lstdefinestyle{mystyle}{
  backgroundcolor=\color{backcolour},   commentstyle=\color{codegreen},
  keywordstyle=\color{magenta},
  numberstyle=\tiny\color{codegray},
  stringstyle=\color{codepurple},
  basicstyle=\footnotesize,
  breakatwhitespace=false,         
  breaklines=true,                 
  captionpos=b,                    
  keepspaces=true,                 
  numbers=left,                    
  numbersep=5pt,                  
  showspaces=false,                
  showstringspaces=false,
  showtabs=false,                  
  tabsize=2
}
\lstset{
language=R,   % R code
literate=
{á}{{\'a}}1
{à}{{\`a}}1
{ã}{{\~a}}1
{é}{{\'e}}1
{ê}{{\^e}}1
{í}{{\'i}}1
{ó}{{\'o}}1
{õ}{{\~o}}1
{ú}{{\'u}}1
{ü}{{\"u}}1
{ç}{{\c{c}}}1
}

%"mystyle" code listing set
\lstset{style=mystyle}
%\bibliography{bibliografia}
\begin{document}

%%
% Definições
%
\def\titulo{The Potatoes Project}
\def\subtitulo{A Types Analyzer Project}
\def\data{28 Junho 2018}
\def\autores{Inês Justo, Luís Moura, Maria Lavoura, Pedro Teixeira}
\def\autorescontactos{inesjusto@ua.pt (84804), luispedromoura@ua.pt (83808)}
\def\autorescontactoss{mjlavoura@ua.pt (84681), pedro.teix@ua.pt (84715)}
\def\versao{Versão Final}
\def\departamento{Departamento de Eletrónica, Telecomunicações e Informática}
\def\empresa{Universidade de Aveiro}
\def\logotipo{images/ua.pdf}
%
%%%%%% CAPA %%%%%%
%
\begin{titlepage}

\begin{center}
%
\vspace*{50mm}
%
{\Huge \titulo}\\ 
%
\vspace{10mm}
%
{\Large \subtitulo}\\
%
\vspace{10mm}
%
{\LARGE \autores}\\ 
%
\vspace{30mm}
%
\begin{figure}[h]
\center
\includegraphics{\logotipo}
\end{figure}
%
\vspace{30mm}
\end{center}
%
\begin{flushright}
\versao\\
\empresa
\end{flushright}
\end{titlepage}

%%  Página de Título %%
\title{%
{\Huge\textbf{\titulo}}\\
{\Large \departamento\\ \empresa}
}
%
\author{%
    \autores \\
    \autorescontactos \\
    \autorescontactoss
}
%
\date{\data}
%
\maketitle
\pagenumbering{roman}

%%%%%%%%%%%%%%%%%%%%%%%%%%%%%%%%%%%%%%%%%%%%%%%%%%%%%%%%%%%%%%%%%%%%%%%%%%%%%%%%%%%%%%%%%%%%%%%%%%%%%%%%%%%%%%%%
% Abstract
\begin{abstract}
Este relatório começa por descrever como se utiliza o projecto desenvolvido. De seguida, descreve as linguagens desenvolvidas (nomeadamente as instruções que suportam) para a criação de um compilador para a linguagem Java de uma linguagem general-purpose cujas operações estão garantidas por um sistema de tipos. Por último são apresentadas algumas considerações sobre a implementação.
\end{abstract}

%%%%%% Agradecimentos %%%%%%
% Segundo glisc deveria aparecer após conclusão...
\renewcommand{\abstractname}{Agradecimentos}
\begin{abstract}
Agradecemos ao prof. Miguel Oliveira e Silva pela enorme ajuda dada e pela paciência a esclarecer as muitas dúvidas que foram surgindo.
\end{abstract}

\renewcommand{\abstractname}{}
\begin{abstract}
"x quê? Batatas?"
\end{abstract}

\tableofcontents
%\listoftables    
%\listoffigures   
\lstlistoflistings

%%%%%%%%%%%%%%%%%%%%%%%%%%%%%%%
\clearpage
\pagenumbering{arabic}

%%%%%%%%%%%%%%%%%%%%%%%%%%%%%%%%%%%%%%%%%%%%%%%%%%%%%%%%%%%%%%%%%%%%%%%%%%%%%%%%%%%%%%%%%%%%%%%%%%%%%%%%%%%%%%%%
% Introdução
\chapter{Introdução}
\label{chap.introducao}

\section{Notas sobre o nome do projecto}
O nome \texttt{Potatoes} é o nome dado quer ao projecto quer à linguagem general purpose desenvolvida e advém da pergunta feita por muitos professores, sobretudo durante os primeiros anos de ensino, quando um aluno não indica as unidades de uma dada operação :\textit{ "x quê? Batatas?"}. 

%Pareceu-nos pois interessante e justificável este nome para uma linguagem associada a um projecto de análise de tipos. %MELHORAR 

\section{Conteúdos do Repositório}
O repositório contém, para além da pasta do relatório, a pasta que contém o código fonte do projecto (\textit{src}):

\begin{itemize}
    \item \textit{compiler}: Contém as classes associadas à compilação e à análise semântica de código fonte da linguagem general purpose (\textit{Potatoes}) para a linguagem Java; 
    \item \textit{potatoesGrammar}: Contém a gramática da linguagem general purpose \textit{Potatoes}; 
    \item \textit{tests}: Contém ficheiros de teste/exemplo de código fonte para ambas as linguagens desenvolvidas; 
    \item \textit{typesGrammar}: Contém a gramática as classes associadas à intepretação de código fonte da linguagem \texttt{Types}; 
    \item \textit{utils}: Contém classes auxiliares. 
\end{itemize}

\section{Como Instalar}
\begin{itemize}
    \item O repositório não contém as classes resultantes à compilação das gramáticas \textit{ANTLR}. Se for a primeira vez que o projecto é utilizado, ou após correr \texttt{antlr4-clean}, é necessário compilar o projecto estando na pasta \textit{src} e utilizando o \textit{script} fornecido:

\texttt{./build}.

    \item Assume-se que estão instalados e adicionados ao \textit{classpath} os JARs associados ao \textit{ANTLR} e \textit{String Template} (disponíveis na pasta \textit{antlr}). Assume-se também que estão instalados os \textit{scripts} disponibilizados (\texttt{antlr4-build}, etc).
\end{itemize}

\pagebreak
\section{Como Utilizar}
\begin{itemize}
     \item Para compilar um programa válido, são necessários 2 ficheiros, um contendo código fonte à linguagem de definição de tipos (por exemplo \textit{tests/global/CompleteExample\_Types.txt}) e um contendo código fonte da linguagem general purpose (por exemplo \textit{tests/global/CompleteExample.txt}).
     \item Por exemplo, para compilar um ficheiro \textit{test.txt}, com um ficheiro de tipos \textit{types.txt} basta:
     \begin{itemize}
        \item Garantir que a primeira linha de código do ficheiro \textit{test.txt} contém a instrução 
        
        \texttt{using "types.txt";}
        \item Compilar executando 
        
        \texttt{./compile text.txt}
        \item Caso não existam erros nos ficheiros a compilar, será gerado um ficheiro chamado \textit{test.txt.java}. 
     \end{itemize}
\end{itemize}

%%%%%%%%%%%%%%%%%%%%%%%%%%%%%%%%%%%%%%%%%%%%%%%%%%%%%%%%%%%%%%%%%%%%%%%%%%%%%%%%%%%%%%%%%%%%%%%%%%%%%%%%%%%%%%%%
% Desenvolvimento
\chapter{Documentação}
\label{chap.documentacao}

% Potatoes
\section{Linguagem de Definição de Tipos}
\label{chap.documentacao.sec.types}
A linguagem de definição de tipos, definida pelas gramáticas do ficheiro \textit{Types.g4}, define as instruções necessárias para a definição de tipos e de prefixos\footnote{Deprecated. Não há análise semântica para o uso de prefixos na linguagem general purpose e nenhum exemplo fornecido para a linguagem general purpose utiliza prefixos.}.

\subsection{Palavras Reservadas}
As palavras reservadas da linguagem são: \textit{types} e \textit{prefixes}.

\subsection{Instruções básicas e comentários}
Todos os ficheiros de código fonte desta linguagem têm obrigatoriamente de conter a instrução:
\begin{lstlisting}[caption= Linguagem de Definição de Tipos - Instrução \texttt{types \{...\} }]
types {
   ... 
}
\end{lstlisting}
podendo ou não conter a instrução:
\begin{lstlisting}[caption= Linguagem de Definição de Tipos - Instrução \texttt{prefixes \{...\} }]
prefixes {
    ...
}
\end{lstlisting}
antes ou depois da instrução \texttt{types}.

\subsubsection{Terminação de instruções}
Nenhuma instrução necessita de um terminador.

\subsubsection{Comentários}
São suportados comentários \textit{in-line}, iniciados pela sequência \texttt{//} (equivalente à linguagem Java). Não são suportados comentários \textit{multiline}.

\pagebreak
\subsection{Definição de Tipos}
Dentro da instrução \texttt{types \{...\}}, podem ser definidos 0 ou mais tipos. Cada tipo pode ser definido através de três instruções distintas:
%%%%%%%%%
\subsubsection{Tipo Básico (Numérico)}
\begin{lstlisting}[caption= Linguagem de Definição de Tipos - Instrução de definição de tipo básico (numérico)]
<ID> "<STRING>" 	
\end{lstlisting}
em que:
\begin{itemize}
    \item \texttt{ID} representa o nome do tipo.
    \item \texttt{STRING} representa o nome de impressão do tipo.
\end{itemize}
\noindent
Por exemplo:
\begin{lstlisting}
meter "m" 	
\end{lstlisting}
representa um tipo chamado \textit{meter}, com nome de impressão \textit{m}.
%%%%%%%%%

\subsubsection{Tipo Derivado (baseado em aritmética entre tipos já-existentes)}
\begin{lstlisting}[caption= Linguagem de Definição de Tipos - Instrução de definição de tipo derivado (baseado em aritmética entre tipos já-existentes)]
<ID> "<STRING>" : <OPERATION BETWEEN PRE-EXISTING TYPES>	
\end{lstlisting}
em que:
\begin{itemize}
    \item \texttt{OPERATION BETWEEN PRE-EXISTING TYPES} representa 1 ou mais multiplicações  (\texttt{typeA * typeB}), divisões (\texttt{typeA / typeB}) ou potências (\texttt{typeA \string^ <INT>}) entre tipos pré-existentes.
\end{itemize}
\noindent
Por exemplo:
\begin{lstlisting}
velocity "v" : distance / time
\end{lstlisting}
representa um tipo chamado \textit{velocity}, com nome de impressão \textit{v}, definida como sendo a divisão dos tipos \textit{distance} e \textit{time}.

\begin{lstlisting}
velocitySquare "v2" : velocity^2
\end{lstlisting}
representa um tipo chamado \textit{velocitySquare}, com nome de impressão \textit{v2}, definida como sendo o quadrado do tipo \textit{velocity}.

%%%%%%%%%
\subsubsection{Tipo Derivado (baseado em conversões entre tipos já-existentes)}
\label{chap.documentacao.derivedTypeOr}
\begin{lstlisting}[caption= Linguagem de Definição de Tipos - Instrução de definição de tipo derivado (baseado em conversões entre tipos já-existentes)]
<ID> "<STRING>" : (<REAL NUMBER>) <TYPE ALTERNATIVE ID> | (<REAL NUMBER>) <TYPE ALTERNATIVE ID> ...
\end{lstlisting}

em que:
\begin{itemize}
    \item o padrão \texttt{| (<REAL NUMBER>) <TYPE>} pode repetir-se 0 ou mais vezes.
    \item \texttt{<REAL VALUE>} representa um número real ou uma operação entre números reais. As operações suportadas são potência, multiplicação, divisão, soma e subtração, existindo também associatividade através do uso de parêntesis. Esse número real \textbf{representa o factor de conversão do tipo \texttt{ID} para o tipo \texttt{TYPE ALTERNATIVE ID}.}
    \item \texttt{<TYPE ALTERNATIVE ID>} representa o nome do tipo alternativa.
\end{itemize}
\noindent
Por exemplo:
\begin{lstlisting}
distance "d" : (-1+ 49725)meter 	| (0.025)inch | (0.9)yard
\end{lstlisting}
representa um tipo chamado \textit{distance}, com nome de impressão \textit{d}, definida como sendo $(-1+ 49725)$ \texttt{meter} ou $(0.025)$ \texttt{inch} ou $(0.9)$ \texttt{yard}.\footnote{Factores de conversão não correspondem à realidade.}. Em linguagem natural, tal é equivalente a dizer que \textit{1 distance é igual a $(-1+ 49725)$ \texttt{meters} ou $(0.025)$ \texttt{inches} ou $(0.9)$ \texttt{yards}}.

%%%%%%%%%%%%%%%%%%%%%%%
\subsection{Definição de Prefixos} 
Dentro da instrução \texttt{prefixes \{...\}}, podem ser definidos 0 ou mais prefixos. Cada prefixo pode ser definido através da seguinte instrução:
\begin{lstlisting}[caption= Linguagem de Definição de Tipos - Instrução de definição de prefixo]
<ID> "<STRING>" : <REAL VALUE>
\end{lstlisting}
em que:
\begin{itemize}
    \item \texttt{ID} representa o nome do prefixo.
    \item \texttt{STRING} representa o nome de impressão do prefixo.
    \item \texttt{REAL VALUE} representa um número real ou uma operação entre números reais. As operações suportadas são as mesmas das referidas na sub-subsecção \ref{chap.documentacao.derivedTypeOr}.
\end{itemize}
\noindent
Por exemplo:
\begin{lstlisting}
DECA  "da"  : 10^1 
\end{lstlisting}
representa um prefixo chamado \textit{DECA}, com nome de impressão \textit{da}, definida como sendo o valor $10^1$.
%%%%%%%%%%%%%%%%%%%%%%%%%%%%%%%%%%%%%%%%%%%%%%%%%%%%%%%%%%%%%%%%%%%%%%%%%%%%%%%%%%%%%%%%%%%%%%%%%%%%%%%%%%%%%%%%
% Potatoes Language 
\pagebreak
\section{Linguagem General-Purpose de Utilização de Tipos}
\label{chap.documentacao.sec.potatoes}
A linguagem general purpose, definida pelas gramáticas do ficheiro \textit{Potatoes.g4}, define as instruções necessárias para operações básicas com variáveis de tipos distintos, cujo universo é definido pelos ficheiros de código-fonte da linguagem de tipos descrita anteriormente.

A maioria das instruções suportadas seguem a sintaxe típica de linguagens como \textit{C} ou Java, pelo que optamos por detalhar apenas as instruções longe dessa sintaxe típica. 

%%%%%%%%%%%%%%%%%%%%%%%%%%%%%%%%%%%%%%%%%%%%%
\subsection{Palavras Reservadas}
As palavras reservadas da linguagem são: \textit{using}, \textit{main}, \textit{fun}, \textit{return}, \textit{if}, \textit{else}, \textit{for}, \textit{while}, \textit{number}, \textit{boolean}, \textit{string}, \textit{void}, \textit{false}, \textit{true}, \textit{print}, \textit{println}.

\subsection{Instruções básicas e comentários}
\subsubsection{Terminação de instruções}
Todas as instruções, à excepção de instruções condicionais e de repetição, são necessariamente terminadas por \texttt{;}.

\subsubsection{Comentários}
São permitidos comentários de linha e de bloco (\textit{multiline}) no formato equivalente à linguagem Java:
\begin{lstlisting}[caption= Linguagem General-Purpose - Comentários]
// commented line
/*
   commented block;
*/
\end{lstlisting}

\subsubsection{Importar um ficheiro de tipos}
O código fonte da linguagem de definição de tipos e o código fonte da linguagem \textit{Potatoes} têm de estar separados em ficheiros diferentes. Existe, com o propósito de importar um ficheiro com diferentes definições de tipos, o comando \texttt{using} seguido de uma String com o \textit{path} para o ficheiro pretendido. Por exemplo:
\begin{lstlisting}[caption= Linguagem General-Purpose - Instrução \texttt{using}]
using "src/TypesDefinitionFiles.txt";
\end{lstlisting}
Esta instrução é \textbf{obrigatória} e tem de constituir a \textbf{primeira instrução} no ficheiro.

\subsection{Instruções condicionais e de repetição}
São permitidas instruções de repetição \texttt{for} e \texttt{while} e condicionais \texttt{if}, \texttt{else if} e \texttt{else} em formato semelhante à linguagem Java.

\subsection{Output}
As funções de impressão estão incluídas na linguagem \textit{Potatoes} com as instruções \texttt{print} e com funcionamento semelhante à linguagem Java, com a limitação de não permitirem operações. 

No caso específico da impressão de variáveis, é impresso o seu valor seguido do seu nome de impressão definido no código fonte da linguagem de definição de tipos.
\begin{lstlisting}[caption= Linguagem General-Purpose - Impressão]
meter m1 = (meter) 5;
println("This code is " + m1 + "long!");    // output: This code is 5 m long!
\end{lstlisting}

\subsection{Funções}
A implementação de funções, apesar de definida deste o início do projeto, não pôde ser implementada. Ainda assim, manteve-se o formato de definição de uma função \textit{"main"}. Assim, após a declaração do ficheiro de definição de tipos é possível fazer declarações de variáveis globais e atribuição de valores e declaração de funções. 

Para criar a função \textit{main} deverá ser usada a assinatura:
\begin{lstlisting}[caption= Linguagem General-Purpose - Assinatura da Função \textit{Main}]
fun main{
    ...
}
\end{lstlisting}
%%%%%%%%%%%%%%%%%%%%%%%%%%%%%%%%%%%
\subsection{Declaração de variáveis}
A linguagem \textit{Potatoes} disponibiliza 3 tipos base além daqueles definidos na Linguagem de Definição de Tipos: \textit{number}, um tipo numérico real adimensional, \textit{boolean} e \textit{string}.

A declaração de variáveis, apesar de apresentar uma sintaxe em tudo semelhante à linguagem Java, tem de ter em conta a compatibilidade entre tipos. Em particular, há 2 situações distintas:

\begin{itemize}

\item \texttt{Inicialização de variável com atribuição dinâmica de valor.}
\begin{lstlisting}
meter m = (meter) 3;
\end{lstlisting}
Uma vez que a gramática de tipos implementa operações com dimensões, a atribuição de um valor numérico adimensional deixa de fazer sentido. Pelo que é sempre necessário fazer o \textit{cast} desse valor para o tipo a inicializar.

\item \texttt{Atribuição de valor usando uma variável de tipo compatível.}
\begin{lstlisting}
meter m = yardVar;
\end{lstlisting}
A compatibilidade entre tipos implica que podem ser considerados como unidades de uma mesma dimensão, pelo que essa atribuição de valor é direta sem \textit{cast} havendo apenas a necessidade de se fazer a conversão do valor numérico com o factor correspondente.

\end{itemize}

\subsection{Operações}
A gramática Potatoes suporta operações numéricas (\%,\^,*,/,+,-) com todos os tipos numéricos, e lógicas com tipos numéricos e booleanos (<,>,<=,>=,==,!=,!) com uso semelhante à linguagem Java mas condicionadas pelo sistema de tipos.

Por exemplo, para as seguintes declarações:
\begin{lstlisting}
meter m = (meter) 1;
yard y = (yard) 2;
\end{lstlisting}

\begin{itemize}

\item \texttt{Adição e Subtração} são permitidas entre tipos numéricos iguais ou compatíveis.
\begin{lstlisting}
meter m2 = m + m;  // correto
meter m3 = m + y;  // correto
meter m4 = m + 1;  // errado (metro e adimensional incompatíveis)
\end{lstlisting}

\item \texttt{Multiplicação e Divisão} são permitidas entre qualquer tipo numérico, sendo que a operação poderá criar unidades novas que estejam ou não definidas inicialmente.
\begin{lstlisting}
meter m2 = m * m;   // operação correta, atribuição errada
meter m3 = m * y;   // operação correta, atribuição errada
number n1 = m / m;  // operação correta e atribuição correta
\end{lstlisting}

\item \texttt{Módulo e Potência} são apenas realizáveis entre um tipo numérico e um valor numérico (ou o tipo \textit{number}).
\begin{lstlisting}[mathescape=true]
meter m2 = m^2;     // operação correta, atribuição errada
meter m3 = m \% 2   // operação correta, atribuição correta
number n1 = m \% m; // operação errada
\end{lstlisting}

\item \texttt{Comparações "$<$, $>$, $<=$, $>=$"} são apenas permitidas entre tipos numéricos compatíveis.
\begin{lstlisting}
boolean b1 = m < m2; // operação permitida
boolean b2 = m >= y; // operação permitida
boolean b3 = m > b1; // operandos não compatíveis
\end{lstlisting}

\item \texttt{Comparações "==, !="} são apenas permitidas exclusivamente entre tipos booleanos ou entre tipos numéricos compatíveis.

\item \texttt{Operações lógicas "\&\&, ||, !"} são permitidas entre tipos booleanos e resultados booleanos de comparações. 
    
\end{itemize}

%%%%%%%%%%%%%%%%%%%%%%%%%%%%%
\chapter{Notas sobre a Implementação}
\label{chap.implementacao}
Embora não seja parte integrante das especificações das linguagens, é relevante apresentar, em traços muitos gerais, a implementação quer do do interpretador da linguagem de tipos quer do analisador semântico da linguagem general purpose.

\section{Interpretador da linguagem de definição de tipos}
\label{chap.implementacao.types}
Os ficheiros com código fonte da linguagem de definição de tipos, após a criação com sucesso da árvore sintática associada a estes, são processados através do seu interpretador, responsável quer pela análise semântica, quer pela interpretação do código. 

A interpretação do código fonte, em ficheiros desta linguagem, implica a criação de estruturas de dados utilizadas depois na análise de tipos feita posteriormente na análise semântica de código fonte da linguagem general purpose (secção \ref{chap.implementacao.potatoes}):
\begin{itemize}
    \item Tabela de Tipos (\texttt{typesTable}) : mapa que corresponde o nome do tipo à sua instância da classe \texttt{Type};
    \item Grafo de Tipos (\texttt{typesGraph}) : grafo em que cada vértice é uma instância da classe \texttt{Type} e cada aresta é uma instância da classe \texttt{Factor}. Por outras palavras, um grafo que representa a ligação entre os tipos e as suas alternativas.
\end{itemize}

\section{Análise semântica da linguagem general purpose}
\label{chap.implementacao.potatoes}
O ponto mais difícil da operação foi o desafio de implementar uma estrutura de métodos capaz de lidar com os tipos compatíveis e todas as suas implicações.
Ao permitir operações que são normalmente fechadas à mesma unidade passa a ser necessário nas multiplicações e divisões além de aceitar qualquer tipo, decidir quando é necessário fazer conversões e quais fazer. Por exemplo para tipos de volume definidos de formas diferentes:
\begin{lstlisting}
metricVolume    "m^3"   : meter * meter * meter
yardVolume      "yd^3"  : yard * yard * yard
weirdVolume     "wV"    : meter * meter * yard
\end{lstlisting}
qualquer operação que envolva 3 variáveis de tipo compatível com os anteriores é válida, no entanto fazendo as conversões corretamente, o valor numérico de volume será diferente na atribuição a cada um dos tipos anteriores.

Deste modo, há duas implementações fundamentais para a resolução correta de operações:

\begin{itemize}
\item \texttt{Códigos de Tipos}

Cada tipo simples ou derivado é identificado com um número primo. Os tipos obtidos através de composição de outros tipos são a composição dos seus códigos únicos. Desta forma qualquer que seja a operação realizada com tipos, dois códigos iguais serão sempre obrigatoriamente identificadores de tipos iguais.
\pagebreak
\item \texttt{Algoritmo \textit{Greedy} de Cálculo}

O algoritmo de cálculo está enraizado na utilização de duas estruturas auxiliares específicas:
\begin{itemize}
    \item A classe \textit{Type} que contêm especificamente um \textit{checklist} de tipos pertencentes à sua composição. 
    \item Um grafo de tipos compatíveis (secção \ref{chap.implementacao.potatoes}) que liga duplamente tipos (os vértices) e contém os fatores de conversão (as arestas) entre eles e a sua relação de herança (não implícita no grafo em si).
\end{itemize}
Por exemplo, para a operação seguinte:
\begin{lstlisting}
weirdVolume = (meter * meter * meter * gram) / ounce;
\end{lstlisting}

Os primeiros passos seriam:
\begin{enumerate}
    \item \texttt{meter * meter}: são marcados na checkList de \textit{weirdvolume} os dois tipos \textit{meter}. Fica por marcar um tipo \textit{yard};
    
    \item\texttt{area * meter} :
        \begin{enumerate}
            \item \texttt{area}
                \begin{enumerate}
                    \item é feita uma tentativa sem sucesso de marcar \textit{area} na checklist de \textit{weirdVolume};
                    \item é feita uma tentativa sem sucesso de converter \textit{area} para o primeiro tipo já marcado;
                    \item é feita uma tentativa sem sucesso de converter \textit{area} para o seu parente hierárquico mais alto;
                    \item a variável é aceite com o seu valor sem conversão e sem alterar o seu tipo;
                \end{enumerate}
            \item \texttt{meter} é feita tentativa com sucesso de marcar meter na checklist de \textit{weirdVolume}.
        \end{enumerate}
\end{enumerate}
Sendo que o algoritmo continua de igual forma e resolve com sucesso qualquer operação com tipos compatíveis com herança simples e vários tipos iguais com herança múltipla. 

No caso de haver tipos diferentes com herança múltipla, para operações e atribuições invulgares, o algoritmo pretende alcançar o resultado correto a maioria das vezes, não sendo garantido o resultado.

Prevemos que a única outra solução envolveria calcular todas as combinações possíveis, que pode facilmente tornar-se impraticável mesmo com exemplos relativamente simples.

\end{itemize}

%%%%%%%%%%%%%%%%%%%%%%%%%%%%%%%%%%%%%%%%%%%%%%%%%%%%%%%%%%%%%%%%%%%%%%%%%%%%%%%%%%%%%%%%%%%%%%%%%%%%%%%%%%%%%%%%
% Conclusão
%\chapter*{Contribuições dos autores}
%\noindent
%\begin{itemize}
% 100% - 500 pontos
%\item Inês Justo: x\% (y pontos)
%\item Luís Moura: x\% (y pontos) 
%\item Maria Lavoura: x\% (y pontos)
%\item Pedro Teixeira: x\% (y pontos)
%\end{itemize}
               
%%%%%%%%%%%%%%%%%%%%%%%%%%%%%%%%%%%%%%%%%%%%%%%%%%%%%%%%
% Acrónimos
%\chapter*{Acrónimos}
%\begin{acronym}
%\end{acronym}
%%%%%%%%%%%%%%%%%%%%%%%%%%%%%%%%%%%%%%%%%%%%%%%%%%%%%%%%
% Bibliografia
%\printbibliography
\end{document}